\documentclass[conference]{IEEEtran}
\usepackage{cite}
\usepackage{amsmath}
\usepackage{algorithmic}
\usepackage{url}
\usepackage{listings}

% Basically, \url{my_url_here}.

% correct bad hyphenation here
\hyphenation{op-tical net-works semi-conduc-tor}

% need about 4 pages full to get 4K words
\begin{document}
%
% paper title
% Titles are generally capitalized except for words such as a, an, and, as,
% at, but, by, for, in, nor, of, on, or, the, to and up, which are usually
% not capitalized unless they are the first or last word of the title.
% Linebreaks \\ can be used within to get better formatting as desired.
% Do not put math or special symbols in the title.
\title{A Study of Identity Based Encryption Systems}

\author{\IEEEauthorblockN{Samuel Petit}
\IEEEauthorblockA{3rd Year Integrated Computer Science student at
Trinity College Dublin, the University of Dublin \\
Email: petits@tcd.ie}
}

% make the title area
\maketitle

% As a general rule, do not put math, special symbols or citations
% in the abstract
\begin{abstract}
The abstract goes here.
\end{abstract}

\section{Introduction}
Will be studying IBE. A type of Asymmetric paired keys 
encryption system. Charcterised with the fact that one of its
keys identifies the recipient (email...phone no).

Makes a lot of sense to compare with classical 
systems currently in use such as RSA.

So let's compare both systems and try to explain why IBE is not as popular

\subsection{Introduction to Public Key Cryptography}
Public Key Cryptography, also called Asymmetric Cryptography is an 
encryption scheme. In other words, it is a method for encrypting messages.
Unlike symmetric cryptography which uses the same key for both encrypting and decrypting
messages, asymmetric cryptography uses two different keys such that one 
is used for encrypting and the other for decrypting information.
In this context, a key is a string of characters that is used in some mathematical
formula to turn a information (a message for example) such that it is impossible to understand in its encrypted form. 
Similarly, we would then also use a key to map the encrypted information back to its original, usable form.
In asymmetric encryption, we typically call the encryption key
the Public Key, and the decryption key the Private Key.
The first asymmetric key cryptosystem was published in 1976 by Whitfield Diffie and Martin Hellman, previously,
all useful modern encryption system used symmetric key encryption systems. 
While both systems still have their usages in todays world,
asymmetric key encryption systems are now used on a daily basis throughout the world. With 
systems such as both HTTP over TLS and HTTP over SSL protocols, digitally signed files, bitcoin,
encrypted messaging services and many other all using some for of asymmetric encryption. 

\subsubsection{How it works}
Put simply, let's say we had a public key $K_{public}$ and $K_{private}$.
We would then obtain a ciphertext $cipher$ with the following formula: 
\begin{equation*}
    cipher = K_{public}(message)
\end{equation*}
Similarly, we obtain the original message from the ciphertext with the following formula:
\begin{equation*}
    message = K_{private}(cipher)
\end{equation*}

\subsection{Different implementations}
There are many different implementations of asymmetric 

\subsubsection{Requirements for public key algorithms (key gen, encrypt and decrypt)}
explain Requirements here

\subsubsection{How RSA is Implemented}

\begin{algorithmic}
\STATE $p \leftarrow prime()$
\COMMENT{prime() returns a prime number}
\STATE $q \leftarrow prime(N)$
\STATE $n \leftarrow p \* q$
\STATE $\phi (n) \leftarrow (p - 1) \* (q - 1)$
\STATE $e \leftarrow coprime(\phi (n))$
\COMMENT{coprime returns a value coprime to $\phi$ where $1 < e < \phi $}
\STATE $d \equiv e^{-1} mod \phi(n)$
\end{algorithmic}
To explain a bit more about this algorithm, we start by generating randomly two prime numbers
p and q. We obtain n from their product, n is used as part of the encryption and decryption function
as we will see soon. Then we use Euler's totient function such as to count the positive integers
up to $(p - 1) \* (q - 1)$, we could alternatively use Carmichael's totient function here 
with a slight change in algorithm, the same keys would be generated regardless. We then pick a value e
such that it is positive, smaller than our value obtained from Euler's totient $\phi(n) $ function and coprime to it.
We have then obtained the public key $\{e,n\}$. To compute our private key with simply compute 
the modular multiplicative inverse of e modulo $\phi(n)$ ($d \equiv e^{-1} mod(\phi(n))$) to obtain
our private key: $\{d,n\}$.
\subsubsection{Explain how IBE works}
explain IBE systems


Given a public key $\{e,n\}$, we can then very simply encrypt a message $M$.
This is done simply by computing:
\begin{equation*}
    C = M^{e}\mod n
\end{equation*}
This part is fairly simple as it only contains a single operation, though keep in mind that
in practice $e$ could be a very large number so $M^{e}$ could be demanding on the device.


Finally, for decryption, given that we have a private key $\{d,n\}$, and a ciphertext
which was encrypted using the private key's corresponding public key, we can obtain the original
message M using a formula very similar to that of the encryption:
\begin{equation*}
    M = C^{d}\mod n
\end{equation*}

\subsection{Math Systems at the core of Public Key Encryption}
Explain that at the core of public key encryption, underlies
many important mathematical concepts.

\subsubsection{Extended euclidean algo to find gcd, coefficients}
euclidean algo

\subsubsection{Fermat's little theorem}
fermats little thm, its place in public crypto

\subsubsection{Prime numbers}
prime numbers at the core of crypto

\subsubsection{Discrete logarithms}
discrete log problem

\subsubsection{Chinese remainder theorem}
chinese remainded thm at the core of decryption (or encryption)

\subsubsection{Square and multiply}
square and multiply to compute large exponents

\subsection{What is identity based encryption ?}
Explain what is exactly IBE now that we know more about public key encryption

\subsubsection{IBE typical implementation}
how is an IBE system implemented

\subsubsection{What are its advantages and flaw?}
pros and cons of IBE

\subsection{IBE vs RSA}
Compare the two

\subsubsection{Attacks on IBE Public key systems}
Go through possible attacks on Public key systems

\subsubsection{Are these attacks all possible on IBE? is it better or worse with IBE?}
go through how IBE protects from attacks differently than rsa systems

\subsubsection{which is most safe?}
expalin which system is safest


% An example of a floating figure using the graphicx package.
% Note that \label must occur AFTER (or within) \caption.
% For figures, \caption should occur after the \includegraphics.
% Note that IEEEtran v1.7 and later has special internal code that
% is designed to preserve the operation of \label within \caption
% even when the captionsoff option is in effect. However, because
% of issues like this, it may be the safest practice to put all your
% \label just after \caption rather than within \caption{}.
%
% Reminder: the "draftcls" or "draftclsnofoot", not "draft", class
% option should be used if it is desired that the figures are to be
% displayed while in draft mode.
%
%\begin{figure}[!t]
%\centering
%\includegraphics[width=2.5in]{myfigure}
% where an .eps filename suffix will be assumed under latex, 
% and a .pdf suffix will be assumed for pdflatex; or what has been declared
% via \DeclareGraphicsExtensions.
%\caption{Simulation results for the network.}
%\label{fig_sim}
%\end{figure}

% Note that the IEEE typically puts floats only at the top, even when this
% results in a large percentage of a column being occupied by floats.


% An example of a double column floating figure using two subfigures.
% (The subfig.sty package must be loaded for this to work.)
% The subfigure \label commands are set within each subfloat command,
% and the \label for the overall figure must come after \caption.
% \hfil is used as a separator to get equal spacing.
% Watch out that the combined width of all the subfigures on a 
% line do not exceed the text width or a line break will occur.
%
%\begin{figure*}[!t]
%\centering
%\subfloat[Case I]{\includegraphics[width=2.5in]{box}%
%\label{fig_first_case}}
%\hfil
%\subfloat[Case II]{\includegraphics[width=2.5in]{box}%
%\label{fig_second_case}}
%\caption{Simulation results for the network.}
%\label{fig_sim}
%\end{figure*}
%
% Note that often IEEE papers with subfigures do not employ subfigure
% captions (using the optional argument to \subfloat[]), but instead will
% reference/describe all of them (a), (b), etc., within the main caption.
% Be aware that for subfig.sty to generate the (a), (b), etc., subfigure
% labels, the optional argument to \subfloat must be present. If a
% subcaption is not desired, just leave its contents blank,
% e.g., \subfloat[].


% An example of a floating table. Note that, for IEEE style tables, the
% \caption command should come BEFORE the table and, given that table
% captions serve much like titles, are usually capitalized except for words
% such as a, an, and, as, at, but, by, for, in, nor, of, on, or, the, to
% and up, which are usually not capitalized unless they are the first or
% last word of the caption. Table text will default to \footnotesize as
% the IEEE normally uses this smaller font for tables.
% The \label must come after \caption as always.
%
%\begin{table}[!t]
%% increase table row spacing, adjust to taste
%\renewcommand{\arraystretch}{1.3}
% if using array.sty, it might be a good idea to tweak the value of
% \extrarowheight as needed to properly center the text within the cells
%\caption{An Example of a Table}
%\label{table_example}
%\centering
%% Some packages, such as MDW tools, offer better commands for making tables
%% than the plain LaTeX2e tabular which is used here.
%\begin{tabular}{|c||c|}
%\hline
%One & Two\\
%\hline
%Three & Four\\
%\hline
%\end{tabular}
%\end{table}



\section{Conclusion}
The conclusion goes here.





% conference papers do not normally have an appendix
\appendices
\section{Proof of the First Zonklar Equation}
Appendix one text goes here.

\section{}
Appendix two text goes here.
\end{document}

